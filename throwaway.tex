\documentclass[10pt]{article}

\usepackage{mcclanahan}
% practice: 1.2.1,2,5,10,20,22,24,25,30,43
% on 43 change "even # of edges" to "even # of 2k edges"
\begin{document}

\section*{Goal}
\textit{(italicized words identify optimization goals)}

Wish to find W which has the \textit{most} regular behavior over the \textit{largest} contiguous subset of initial concentrations $C_0$, given gaussian disturbances of $W$. The contiguity may not be required, but seems to make sense (modelling a region which the cell can reliably keep the ambient concentrations within). We could take into account non-contiguous subsets, but that is not likely to be useful (it would be weird to have a motif which has similar behavior over different non-contiguous regions of ambient concentration).

\section*{Formulation}
We define the norm over a matrix.

\begin{equation}
\lvert W \rvert: \mathbb{R}_{n \times n} \to \mathbb{R} = \sqrt{\sum\limits_{i, j \in [1,n]} W_{ij}}
\end{equation}

Given fixed parameters $W \in \mathbb{R}_{n \times n}, \sigma \in \mathbb{R}$, define a Gaussian distribution of weights.

\begin{equation}
  D(W'|W,\sigma) = (2 \sigma^2 \pi)^{-2} e^\frac{-(\lvert W' - W \rvert)^2}{2 \sigma^2}
\end{equation}

This says ``mutations modify the weights of the matrix $W'$ a Euclidean distance from the original weight matrix $W$ with probability according to the gaussian distribution with parameter $\sigma$.'' This may not be realistic, but it captures the idea that if a mutation occurs, it is likely to modify multiple weights at once if it modifies any at all.

We define the ``behavior'' of a network $(C_0, W)$ as the concentration vector $C(T)$ at some given time $T$. This assumes linear differential equations modelling activation and inhibition rates with no external input or output.

\begin{equation}
  C(t|C_0, W) = C_0 + \int\limits_0^t W \cdot C(t) dt
\end{equation}

This may not be appropriate; some networks may have \textit{very} reliable periodic behavior, where the period simply changes slightly in response to disturbance, which would lead to vastly different $C(T)$. However, if we restrict ourselves to networks which reach some sort of equilibrium (which need not be defined or used at this time), we can use just $C(T)$. We define networks that reach equilibrium over some region $\mathcal{C}$ as networks which have similar $C(T)$ for some large $T$ over $\mathcal{C}$. $T > 0$ and must be ``large enough to allow the system to get to equilibrium.'' We can accomplish this by simply choosing a timescale.

\subsection*{Variability}
Variability $V$ of a network is defined using variation $v$ with an initial concentration $(C_0, W)$ over some stable $\mathcal{C}$ is defined. $\mathcal{C}$ is the aforementioned contiguous stable region of $C_0 \in \mathbb{R}^n$.

\begin{align}
  \mathcal{C}(W) &\in \mathbb{R}^n \\
  v(C_0, W', W) &= \int\limits_{C'_0 \in \,\mathcal{C}(W)} \lvert C'_0 - C_0 \rvert \cdot \lvert C(T|C'_0, W') - C(T|C_0, W) \rvert \\
  V(C_0, W) &= \int\limits_{W' \in \mathcal{W}}D(W'|W, \sigma) \cdot v(C_0, W', W)
\end{align}

This is not what we want, even if it's the right idea -- how do we find $\mathcal{C}(W)$ given $W$? Also, we want to make this a continuous measure over all $C_0$, not just those in $\mathcal{C}$, while maintaining the original objective. \textbf{This is what I'm currently stuck on.}

Finding stable networks $X$, then, is defined as an optimization problem. The most robust networks (the ones with the minimum variability) are (hopefully) likely to produce interesting network motifs.

\begin{equation}
  X^* = \min\limits_W \int\limits_{C_0 \in \mathcal{C}(W)}V(C_0, W)
\end{equation}

Simulation seems possible when the above issues are addressed.
\end{document}
