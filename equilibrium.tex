\documentclass{article}

\usepackage{amsmath}
\usepackage{amssymb}
\usepackage{hyperref}

\DeclareMathOperator*{\argmin}{argmin}

\title{Topological Modelling of Biological Interactions for Robustness}
\author{Danny McClanahan and William Cox \\
Vanderbilt University}
\date{}

\begin{document}
\maketitle
\vspace{-25pt}

\section{Goals}
\begin{description}
\item[Notation] \hfill \\
  To establish a framework for understanding chains of biological reactions as dynamical systems.
\item[Analysis] \hfill \\
  To define measures of similarity and robustness for dynamical systems so that a quantifiably small change in the input does not strongly affect the output.
\item[Application] \hfill
  \begin{enumerate}
  \item To create a method of finding systems which produce the desired chemical concentrations with a high degree of robustness against changes in input.
  \item To find which neighborhoods of inputs to systems lead to groups of systems with similar behavior.
  \item To produce a software library to find robust alternatives to known biological reactions.
  \item To produce a predictor for equilibrium points for these systems.
  \item To use the predictor to identify how different input values can quantifiably affect the output behavior of a system.
  \end{enumerate}
\item[Verification] \hfill
  \begin{enumerate}
  \item To verify that the robustness metric defined in this study increases the resilience of a system to factors such as mutation which can modify biological networks.
  \end{enumerate}
\end{description}

\section{Description as Dynamical System} \label{description}
Biological chemical networks can be described in terms of dynamical systems with inputs. $n$ is the number of chemicals in the system. $W$ is a matrix of functions which map time and concentrations of two chemicals to the amount which the concentration of the first chemical is affected. $W$ models internal influences, such as repression or induction of one chemical on another (or itself). $D$ is a vector of functions which map a time $t$ to some real number. $D$ models external influences, such as an input or degradation of a chemical into the system over time.

Let $\mathbb{R}^+ = \{r \in \mathbb{R} \,\,|\,\, r \ge 0\}$. Formally, we can describe a system $j = \langle s_0, D, W \rangle$ such that $s_0 \in \mathbb{R}^n, D \in (\mathbb{R}^+ \to \mathbb{R})^n, W \in (\mathbb{R}^+ \times \mathbb{R} \times \mathbb{R} \to \mathbb{R})^{n \times n}$. $s(t)$ is a state vector of chemical concentrations at time $t$. Let $s_i(t)$ be the $i^{th}$ component of state vector $x$ at time $t$. $Y$ refers to the set of all such systems $y$. Then, $Y = \mathbb{R}^n \times (\mathbb{R}^+ \to \mathbb{R})^n \times (\mathbb{R}^+ \times \mathbb{R} \times \mathbb{R} \to \mathbb{R})^{n \times n}$.

\begin{align}
  y &= \langle s_0, D, W \rangle \label{system-defn} \\
  \frac{ds_i}{dt}(t) &= D_i(t) + \sum_{j = 1}^n W_{ij}(t, s_j(t), s_i(t)) \label{system-diffeq}
\end{align}

Note that the system is described in terms of $s_0$ and parameters to the differential equation $\frac{ds}{dt}$, not by a single function $s(t)$, its state vector at time $t$. This is because while explicit, closed-form solutions to some dynamical systems (strictly linear ones) do exist, this is not true in the general case, so obtaining a single formula for $s(t)$ is impossible. Modelling of the dynamics of these systems is possible by discretization, but there is some error between the simulated version of the system and the true $s(t)$, which can become extremely significant in some cases, particularly where the dynamics become chaotic. The effect of this discrepancy is discussed in \textbf{(where?)} when applying this model to real systems.

\section{Difference Between Dynamical Systems} \label{similarity}
We define a ``difference'' of dynamical systems of this type to quantify the ``closeness'' of one network's behavior to another. This is defined in two ways. First note that there is a difference between the parameters (``input'') of a system and the system's behavior (``output'').

The inputs of a system are the values of $\langle s_0, D, W \rangle$. The ``distance'' between these is problem-dependent. For modelling a linear system, for example, $D$ and $W$ can be a vector and a matrix, and distance can simply be the euclidean distance, as discussed in \textbf{(where?)}. For more complex systems, ``input distance'' can be mapped to a normal distribution, as discussed in \textbf{(where?)}.

The ``output distance'' of a system is some way to quantify the distance between attractors of the system, or the behavior of the system over time. For systems with a defined convergence point, the distance between the two can be reduced to just the distance between the convergence points, as discussed in \textbf{(where?)}. For more complex periodic or aperiodic systems, other methods are usable as well, as discussed in \textbf{(where?)}.

\subsection{Equilibrium System}
We only concern ourselves with systems ``spinning'' about an ``equilibrium region'' for now. We create a time $t_e : Y \to \mathbb{R}$ ($e$ for ``equilibrium'', even though actual equilibrium is constant) to represent the time at which a system begins spinning about an equilibrium region.

Given some $\epsilon > 0$, and $t_f$ a final time point at which we stop observing the system, we define $t_e(y)$ as follows:

\begin{equation}
  \int_{t_e(y)}^{t_f} \lvert s(t) - s(t_e(y)) \rvert\, dt < \epsilon \label{t_e}
\end{equation}

\textbf{EXPLAIN THAT $t_f$ IS REQUIRED SO THAT THE INTEGRAL CONVERGES}

\subsubsection{Definition of Difference}
For two systems $y_1, y_2 \in Y$, we define the distance $M$ ($M$ for ``metric'', even though this isn't a metric (I think)). For equilibrium systems, we denote this $M_{eq}$. $s_e$ is the average of all points past $t_e$, and represents the equilibrium point of the system.

\begin{align}
  s_e &: Y \to \mathbb{R}^n \\
  s_{e_i}(y) &= \frac{\int_{t_e(y)}^{t_f} s_i(t) \,dt}{t_f - t_e(y)} \,\forall\, i \\
  M &: Y \times Y \to \mathbb{R}^+ \\
  M_{eq}(y_1, y_2) &= \lvert s_e(y_1) - s_e(y_2) \rvert \label{distance}
\end{align}

\subsection{Periodic System}
\textbf{DO AN EXAMPLE OF A PERIODIC SYSTEM AS WELL \\
  ALLOW SHIFTING AND STRETCHING}

\section{Robustness of a Dynamical System} \label{robustness}
``Robustness'' is meant to quantify the tendency of the system to remain similar (the ``similarity'' defined in \ref{similarity}) given small changes in input. Robustness $R$ is described as the inverse of a fragility $F$. There is a simple formulation $F_s$ used to motivate a more general formulation $F_g$.

\begin{align}
  F : Y \to \mathbb{R}^+ \\
  R : Y \to \mathbb{R}^+ \\
  R(y) = F(y)^{-1}
\end{align}

\subsection{Simple Robustness}
The simple formulation quantifies the condition that all ``nearby'' systems have ``similar'' behavior. This is extremely similar to the definition of continuity of a function at a point in a topological space. Nearby systems to a state are those within the ball centered at that state, for a given radius $\epsilon > 0$, using the euclidean metric for \textit{initial state vectors}; this metric is denoted as $m$, separate from the $M$ defined in \eqref{distance}. $s_{i_0}$ refers to the initial state vector $s_0$ for the system $y_i$.
p
The below definition of fragility describes the radius $\delta$ (defined by the difference $M$ in \eqref{distance}) required to contain all systems within radius $\epsilon$ (defined by the metric $m$ described above).

\begin{align}
  m(y_1, y_2) &= \lvert s_{1_0} - s_{2_0} \rvert \label{example-metric} \\
  F_s(y) &= \argmin_{\delta > 0} M(y, y') < \delta \,\forall\, y' \in B_m(s, \epsilon)
\end{align}

\subsection{General Robustness} \label{general-robustness}
The general formulation of fragility of a system is defined in terms of three functions $f$, $g$, and $h$. $f$ ``weights'' each possible pair of initial states, $g$ ``weights'' the difference between each nearby system's behavior (and therefore typically makes use of the distance $M$ from \eqref{distance}), and $h$ combines the two measures.

\begin{align}
  f &: Y \times Y \to \mathbb{R} \\
  g &: Y \times Y \to \mathbb{R} \\
  h &: \mathbb{R} \times \mathbb{R} \to \mathbb{R} \\
  F_g(y) &= \int_{y' \in Y} h(f(y, y'), g(y, y')) \label{general-fragility}
\end{align}

As an example of the utility of $F_g$, it can be used to model the simple formulation $F_s$, given some $\epsilon > 0$.

\begin{align}
  M_{max}(y) &= \max\limits_{\sigma \,s.t.\, m(s_0, \sigma) < \epsilon} M(y, \sigma) \\
  f(y, y') &= 1 \\
  g(y, y') &= \begin{cases}
    1 & M(y, y') < M_{max}(y) \\
    0 & else
  \end{cases} \\
  h(a, b) &= a \cdot b
\end{align}

These general formulations $F_g$ and $R_g$ become extremely useful when characterizing complex or stochastic behavior, as seen in \ref{max-robust}.

\section{Finding Maximally Robust Similar Systems} \label{max-robust}
Given \eqref{system-defn}, \eqref{system-diffeq}, \eqref{distance}, and \eqref{general-fragility}, we can attempt to solve the problem posed in the introduction. We wish to find a system $y^*$ which is similar to some system $y$ and which is maximally robust for some expected values of inputs.

Given $\epsilon > 0$. Let $y' = \langle s_0', D, W \rangle$ for some $s_0' \in \mathbb{R}^n$. This means $y'$ has the same parameters $D$ and $W$ as a network $y \in Y$, but a different initial state.

\textbf{THIS ASSUMES ONLY EQUILIBRIUM WITH $M = \lvert \cdot \rvert$ FIX THAT}

\begin{align}
  Y_{similar} &= \{y' \in Y \,\,|\, M(y, y') < \epsilon \} \\
  y^* &= \argmin_{y' \in Y} F(y')
\end{align}

\subsection{Modelling Stochastic Distributions Using a Weight Function}
In \ref{general-robustness}, we described how we use the function $f$ to ``weight'' certain pairings of systems, in a sort of distance-like metric. This can be used to model stochastic distributions of parameters.

Recall that a system $y \in Y$ is described by a tuple $\langle s_0, D, W \rangle$. We shall assume that each parameter arises from a normal distribution with mean $\mu$ and standard deviation $\sigma$. This produces vectors $\mu_y \in \mathbb{R}^{2n + n^2}, \sigma_y \in \mathbb{R}^{2n + n^2}$ ($2n + n^2$ is the number of all parameters in a system $y$, as described in \ref{description}). We can then make an $f$ which weights other systems $y'$ according to how far they deviate from their mean in $\mu_y$.



\end{document}
